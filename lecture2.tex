\section{Исчисление высказываний}

Матлогика "--- это наука о правильных математических рассуждениях, а поскольку
рассуждения обычно ведутся на каком-то языке, то она неразрывна связана с идеей
двух языков: \emph{языка исследователя} (или иначе его называют \emph{мета-языком}),
и \emph{предметного языка}. Как следует из названий, языком исследователя 
пользуемся мы, формулируя утверждения или доказывая теоремы о разных способах
математических рассуждений, или просто их обсуждая. Сами же математические рассуждения,
собственно и составляющие предмет исследования, формализованы в некотором предметном 
языке.

Мы начнём с очень простого предметного языка "--- языка исчисления высказываний.
Элементами (строками) данного языка являются некоторые выражения (формулы), по структуре
очень похожие на арифметические, которые называются \emph{высказываниями}.

Каждое высказывание "--- это либо \emph{пропозициональная переменная} "--- 
большая буква латинского алфавита, возможно, с цифровым индексом, либо 
оно составлено из одного или двух высказываний меньшего размера, соединённых логической связкой.

Связок в языке мы определим 4 (хотя при необходимости этот список может быть
в любой момент изменен):
\begin{itemize}
\item конъюнкция: если $\alpha$ и $\beta$ "--- высказывания, то ($\alpha \with
  \beta$), ($\alpha \land \beta$) "--- тоже высказывание;
\item дизъюнкция: если $\alpha$ и $\beta$ "--- высказывания, то ($\alpha \vee
  \beta$) "--- тоже высказывание;
\item импликация: если $\alpha$ и $\beta$ "--- высказывания, то ($\alpha
  \rightarrow \beta$), ($\alpha \supset \beta$) "--- тоже высказывание;
\item отрицание: если $\alpha$ "--- высказывание, то $\neg\alpha$ "--- тоже высказывание.
\end{itemize}

При этом $\alpha$ и $\beta$ называются мета-переменными, они являются частью
мета-языка ИВ. В нем мета-переменные для пропозициональных переменных
обозначаются заглавными последними буквами латинского алфавита, а другие
мета-переменные, являющиеся высказываниями "--- строчными начальными буквами греческого алфавита. В мета-языке
запись короче и удобнее.

В отличие от мета-языка, в языке исчисления высказываний важно расставление
скобок, указывающих на приоритет выполнения операций, наиболее приоритетные
операции помещаются в скобки, а так же сохраняется их ассоциативность.

При переходе к мета-языку, есть возможность опускать некоторые скобки, приоритет
операций которых задается самой логической связкой. Такие сокращения действуют по следующим
правилам:
\begin{enumerate}
  \item Приоритет (в порядке уменьшения) : $\neg$, $\with$, $\vee$, $\rightarrow$;
  \item Ассоциативность:
    \begin{itemize}
      \item $\with$, $\vee$ "--- левоассоциативны;
      \item $\rightarrow$ "--- правоассоциативна.
    \end{itemize}
\end{enumerate}
  
Примером может послужить высказывание:
\[\alpha\to(\alpha\to((((\neg\beta)\vee(\beta\with(\neg\alpha)))\vee(\neg\beta))\to((\alpha\with\alpha)\to((\alpha\vee\beta)\vee\beta))))\]
  
Которое на мета-языке записывается следующим образом:
\[\alpha\to\alpha\to\neg\beta\vee\beta\with\neg\alpha\vee\neg\beta\to\alpha\with\alpha\to\alpha\vee\beta\vee\beta\]


\subsection{Оценка высказываний}

Процесс <<вычисления>> значения высказываний (\emph{оценка} высказываний) имеет совершенно
естественное определение. Мы фиксируем некоторое множество
\emph{истинностных значений} $V$, для начала мы в качестве такого множества возьмем 
множество $\{\texttt{И}, \texttt{Л}\}$, здесь \texttt{И} означает истину, а
\texttt{Л} "--- ложь. Всем пропозициональным переменным мы приписываем некоторое
значение, а затем рекурсивно вычисляем значение выражения естественным для указанных
связок образом.

В дальнейшем мы будем брать необычные множества истинностных значений, и будем давать
неожиданный смысл связкам, однако, классическая интерпретация связок всегда будет
подразумеваться, если не указано иного.

Оценку высказываний мы будем записывать с помощью двойных квадратных скобок. Например,
нетрудно видеть, что $\llbracket P \rightarrow P \rrbracket = \texttt{И}$.
Если нам требуется явно задать значения некоторых пропозициональных переменных, мы будем
записывать эти значения как верхний индекс: $\llbracket P \rightarrow Q
\rrbracket ^ {P\coloneqq \texttt{Л}} = \texttt{И}$.

\begin{definition}[тавтология] \emph{Тавтологией} или \emph{общезначимым высказыванием}
  называют такое высказывание $\alpha$, у которого $\llbracket \alpha \rrbracket
  = \texttt{И}$ при любой оценке пропозициональных переменных. Также, на языке исследователя общезначимость высказывания $\alpha$ можно кратко записать как $\models \alpha$. 
\end{definition}

\begin{definition}[следование].
Если $\gamma_1 \dots \gamma_n : \llbracket \gamma_1 \rrbracket = \texttt{И} \dots
\llbracket \gamma_n \rrbracket = \texttt{И}$ влечет $\llbracket \alpha
\rrbracket = \texttt{И}$,  тогда $\alpha$ \emph{следует} из $\gamma_1 \dots
\gamma_n$ (обозначается $\gamma_1 \dots \gamma_n \models \alpha$) 
\end{definition}

\begin{definition}[выполнимость] Формула \emph{выполнимая}, если есть значения
  пропозициональных переменных, когда ее оценка истинна. \emph{Невыполнимая}
  "--- наоборот.
\end{definition}


\subsection{Доказательства}

В любой теории есть некоторые утверждения (аксиомы), которые принимаются без доказательства.
В исчислении высказываний мы должны явно определить список всех возможных аксиом. 
Например, мы можем взять утверждение $Z \with X \rightarrow Z$ в качестве аксиомы.
Однако, есть множество аналогичных утверждений, например, $X \with Z \rightarrow X$,
которые не отличаясь по сути, отличаются по записи, и формально говоря, являются другими
утверждениями.

Для решения вопроса мы введём понятие \emph{схемы аксиом} "--- некоторого обобщённого
шаблона, подставляя значения в который, мы получаем различные, но схожие аксиомы. 
Например, схема аксиом $\alpha \with \beta \rightarrow \alpha$ позволяет получить как
аксиому $Z \with X \rightarrow Z$ (при подстановке $\alpha \coloneqq  Z, \beta
\coloneqq  X$), так и аксиому $X \with Z \rightarrow X$.

Возьмем следующие схемы аксиом для исчисления высказываний.

\begin{tabular}{ll}
(1) & $\alpha \rightarrow \beta \rightarrow \alpha$ \\
(2) & $(\alpha \rightarrow \beta) \rightarrow (\alpha \rightarrow \beta \rightarrow \gamma) \rightarrow (\alpha \rightarrow \gamma)$ \\
(3) & $\alpha \with \beta \rightarrow \alpha$\\
(4) & $\alpha \with \beta \rightarrow \beta$\\
(5) & $\alpha \rightarrow \beta \rightarrow \alpha \with \beta$\\
(6) & $\alpha \rightarrow \alpha \vee \beta$\\
(7) & $\beta \rightarrow \alpha \vee \beta$\\
(8) & $(\alpha \rightarrow \gamma) \rightarrow (\beta \rightarrow \gamma) \rightarrow (\alpha \vee \beta \rightarrow \gamma)$\\
(9) & $(\alpha \rightarrow \beta) \rightarrow (\alpha \rightarrow \neg \beta) \rightarrow (\neg \alpha)$\\
(10) & $\neg \neg \alpha \rightarrow \alpha$
\end{tabular}

Помимо аксиом, нам требуется каким-то образом научиться преобразовывать одни верные утверждения
в другие.
Сделаем это с помощью правил вывода. У нас пока будет одно правило вывода "--- Modus Ponens.
Это также схема, она позволяет при доказанности двух формул $\alpha$ и $\alpha \rightarrow \beta$
считать доказанной формулу $\beta$. Традиционно правило вывода Modus Ponens записывают так:
\[\infer{\beta}{\alpha & \alpha \rightarrow \beta}\]

\begin{definition}[доказательство] \emph{Доказательство} в исчислении высказываний "--- 
это некоторая конечная последовательность выражений 
$\alpha_1$, $\alpha_2$ \dots $\alpha_n$
из языка $L$, такая, что каждое из утверждений $\alpha_i (1 \le i \le n)$
либо является аксиомой, либо получается из других
утверждений $\alpha_{P_1}$, $\alpha_{P_2}$ \dots $\alpha_{P_k}$ 
($P_1 \dots P_k < i$) по правилу вывода.
\end{definition}

\begin{definition}[доказуемость] Высказывание $\alpha$ называется доказуемым, если 
существует доказательство $\alpha_1$, $\alpha_2$ \dots $\alpha_k$, и в нем
$\alpha_k$ совпадает с $\alpha$. 
\end{definition}

Вообще, схемы аксиом и правила вывода существуют для удобства задания
исчисления. В дальнейшем будет очень неудобно возиться с этими объектами.
Поэтому мы считаем, что в исчислении имеется бесконечно много аксиом и правил вывода,
которые порождаются подстановкой всех возможных формул вместо параметров в схемы.

В качестве сокращения записи в языке исследователя мы будем писать $\vdash \alpha$,
чтобы сказать, что $\alpha$ доказуемо.

Соглашение об обозначениях. Будем обозначать буквами 
$\Gamma, \Delta, \Sigma, \Pi$ списки формул (возможно, пустые).

\begin{definition}[вывод из допущений]
Пусть $\Gamma$ "--- некоторый список высказываний, а $\alpha$ "--- 
некоторое высказывание. 
Тогда мы будем говорить, что высказывание $\alpha$ \emph{выводимо} из $\Gamma$ 
(и записывать это как $\Gamma \vdash \alpha$), если существует такая 
последовательность высказываний $\alpha_1, \alpha_2, \dots \alpha_{n-1}, \alpha$
(называемая \emph{выводом} $\alpha$ из $\Gamma$), 
что каждое из высказываний $\alpha_i$ "--- это 
\begin{itemize}
\item либо аксиома,

\item либо получается по правилу Modus Ponens из предыдущих высказываний, 
\item либо "--- высказывание из списка $\Gamma$.
\end{itemize}
Элементы $\Gamma$ мы будем называть \emph{допущениями}. Также эти элементы
называют предположениями или гипотезами. 
\end{definition}

В свете данного определения можно заметить, что доказательство "--- это
вывод из пустого списка допущений. 

Для примера докажем следующую лемму, которая понадобится нам в будущем:

\begin{lemma}
$\vdash \alpha \rightarrow \alpha$
\end{lemma}

\begin{proof}\ 

\begin{tabular}{lll}
(1) & $\alpha \rightarrow (\alpha \rightarrow \alpha)$&Сх. акс. 1\\
(2) & $(\alpha \rightarrow (\alpha \rightarrow \alpha)) \rightarrow 
  (\alpha \rightarrow ((\alpha \rightarrow \alpha) \rightarrow \alpha)) \rightarrow
  (\alpha \rightarrow \alpha)$&Сх. акс. 2\\
(3) & $(\alpha \rightarrow ((\alpha \rightarrow \alpha) \rightarrow \alpha)) \rightarrow
  (\alpha \rightarrow \alpha)$&M.P. 1,2\\
(4) & $(\alpha \rightarrow ((\alpha \rightarrow \alpha) \rightarrow \alpha))$ & Сх. акс. 1\\
(5) & $\alpha \rightarrow \alpha$ & M.P. 4,3\\
\end{tabular}\ 

\end{proof}

И приведем пример доказательства с контекстом:
\begin{lemma}
  $\beta \vdash \alpha \to \alpha \with \beta$
\end{lemma}

\begin{proof}\

\begin{tabular}{lll}
(1) & $\beta$ & Гипотеза 1 \\
(2) & $ \beta \to \alpha \to \beta $ & Сх. акс. 1 \\
(3) & $ \alpha \to \beta $ &M.P. 2,1 \\
(4) & $ \alpha \to \beta \to \alpha \with \beta $ &Cх. акс. 5 \\
(5) & $(\alpha \to \beta) \to (\alpha \to \beta \to \alpha \with \beta) \to \alpha \to \alpha \with \beta $ &Сх. акс. 2\\
(6) & $(\alpha \to \beta \to \alpha \with \beta) \to \alpha \to \alpha \with \beta $ & M.P. 5,3\\
(7) & $\alpha \to \alpha \with \beta $ &M.P. 6,4\\
\end{tabular}\

\end{proof}