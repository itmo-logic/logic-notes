\section{Решётки}
\author{ruperson}
%Интуиционистская логика --- в душе (c) ДГ
В интуиционистской логике привычная 10 аксиома заменяется на это:
на $\alpha \rightarrow \neg \alpha \rightarrow \beta$.
В целом, встав на путь вопроса интуиционизма, мы могли бы пойти и дальше, объявив войну 9-ой аксиоме, которая позволяет доказывать от противного. Декларируем доказательства от противного злом и требуем конкретного построения или идите лесом. Но этого делать мы не будем.
Вернемся к нашим новоизменениям. Что изменилось, какие утверждения пропали? Корректность, очевидно, на месте, так как эта формула была доказуема в классической логике. Пропала полнота.
В интуиционистской логике (в отличие от классической) отсутствует доказуемость этого утверждения $\alpha \vee \neg \alpha $.
Как доказать то, что это высказывание недоказуемо?:  Можно перебрать все доказательства, которых бесконечное количество, как-то подумать и понять это. Сложно. Давайте воспользуемся так называемым <<модельным доказательством>>. Мы предложим такой способ оценки высказываний в интуиционистской логике, при этом что чтобы но позволял оценивать совершенно спокойно оценивать по всем аксиомам логики, при этом доказательство не будет выполнено. Для этого введём следующее понятие:

\begin{definition} Частично упорядоченное множество $\langle{}A, \sqsubseteq  \rangle$ "---  \emph{решётка}, если:
	
	\begin{enumerate}
		\item $\forall a,b \in A ~ \exists$ наименьший $c: a \sqsubseteq c$ и $b \sqsubseteq c $ (обозначение $a+b$);
		\item $\forall a,b \in A ~ \exists$ наибольший $c: c \sqsubseteq a$ и $c \sqsubseteq b$ (обозначение $a \cdot b$).
	\end{enumerate}
Эти $c$ называют точными верхними и нижними гранями соответственно.
\end{definition}
Заметим, что минимальный (максимальный) "--- это не одно и то же, что наименьший (наибольший). В первом случае предполагается, что элементы могут быть несравнимы, а в нашей ситуации все элементы сравнимы между собой, <<настоящий минимум>>. 

Жизненный пример: 2 этажа в доме. Определим, что один этаж больше другого, если он над ним. Есть наибольший этаж в каждом доме, но нет наибольшего этажа для всего мира.


Пример решетки: $\langle{} \mathbb{R}, \leq \rangle$
\begin{gather*}
a + b = \max (a,b) \\
a \cdot b = \min (a,b) \\
\end{gather*}



Пример множества, не являющегося решёткой:
\begin{figure}[ht]
	\centering
	\begin{tikzpicture}[->,>={Stealth[black]}, every edge/.style={draw=black,thick}]
	\node at (0,   0) (A)  {$a$};
	\node at (1.5, -2) (B)  {$b$};
	\node at (-1.5, -2) (C)  {$c$};
	
	\path (A)  edge (B) (A)  edge (C);
	\end{tikzpicture}
	\caption{$a  \sqsubseteq b $ обозначается стрелочкой из $a$ в $b$} 
	\label{not-a-lattice}
	Это множество не является решёткой, потому что для $b$ и $c$ не выполняется второй пункт из определения решётки.
\end{figure}

Ещё пример:\\
$X$ "--- множество множеств какого-то множества $T$.
$a \sqsubseteq b , a,b \in X$, если $a \subseteq b $. Это решётка:
\begin{gather*}
a + b = a \cup b \\
a \cdot b = a \cap b 
\end{gather*}


\begin{definition} \emph{Дистрибутивная решётка} "--- это  $\langle{}X, \sqsubseteq  \rangle$, в которой дополнительно выполнено следующее свойство:
	\[\forall a,b,c  \quad a+b \cdot c = (a+b)\cdot(a+c)\] 
	
\end{definition}

\begin{lemma} В дистрибутивной решетке так же выполнено $a \cdot (b+c) = ab + ac$.
\end{lemma}

\begin{figure}[ht]
	\centering
	\begin{tikzpicture}[->,>={Stealth[black]}, every edge/.style={draw=black,thick}]
	\node at (0,   0) (A)  {$a$};
	\node at (-2, -2) (B)  {$b$};
	\node at (2, -2) (C)  {$c$};
	\node at (0, -4) (D)  {$d$};
	
	\path 
	(A)  edge (B) 
	(A)  edge (C)
	(B)  edge (D)
	(C)  edge (D);
	\end{tikzpicture}
	\caption{Пример дистрибутивной решетки}
	\label{distr-lattice}
\end{figure}

Но гораздо интереснее понять, когда множество не является решеткой.

\begin{theorem}
	Решётка дистрибутивна, когда она не содержит диамантов и пентагонов (рисунок \ref{diamant} и \ref{pentagon}).
	% Возможно избыточное пояснение "Не содержит" имеется в виду выделение этих 5 вершин как подграфа, в котором нет других отношений, кроме стрелочек
\end{theorem}



\begin{figure} [ht]
	\centering
	\begin{minipage}{0.45\textwidth}
		\centering
		\begin{tikzpicture}[->,>={Stealth[black]}, every edge/.style={draw=black,thick}]
		\node at (-0.5,   2) (A)  {$a$};
		\node at (-1.5, 0) (B)  {$b$};
		\node at (0, 0) (C)  {$c$};
		\node at (1.5, 0.3) (D)  {$d$};
		\node at (0, -2.5) (E)  {$e$};
		
		\path 
		(A)  edge (B) 
		(A)  edge (C)
		(A)  edge (D)
		(B)  edge (E)
		(C)  edge (E)
		(D)  edge (E);
		\end{tikzpicture}
		\caption{Фигура \emph{диамант}}
		\label{diamant}
	\end{minipage}\hfill
	\begin{minipage}{0.45\textwidth}
		\centering
		\begin{tikzpicture}[->,>={Stealth[black]}, every edge/.style={draw=black,thick}]
		\node at (0,   0) (A)  {$a$};
		\node at (-2, -2) (B)  {$b$};
		\node at (2, -1.8) (C)  {$c$};
		\node at (2, -4) (D)  {$d$};
		\node at (0.1, -6) (E)  {$e$};
		
		\path 
		(A)  edge (B) 
		(A)  edge (C)
		(C)  edge (D)
		(B)  edge (E)
		(B)  edge (E)
		(D)  edge (E);
		\end{tikzpicture}
		\caption{Фигура \emph{пентагон}}
		\label{pentagon}
	\end{minipage}
\end{figure}

\begin{definition}
	\emph{Импликативная решётка } "--- это дистрибутивная решётка, в которой определено псевдодополнение. 
	
	$c = a \to b$ "--- \emph{псевдодополнение} (не путать с импликацией), если $c=\max \{x \mid x \cdot a \sqsubseteq b \}$, где $\max$ "--- взятие наибольшего элемента.
\end{definition}

Рассмотрим вот такую решетку (рисунок \ref{wise-example}) и посчитаем в ней всякое разное:
\begin{figure}[ht]
	\centering
	\begin{tikzpicture}[->,>={Stealth[black]}, every edge/.style={draw=black,thick}]
	\node at (0,   0) (A)  {$1$};
	\node at (-2, -2) (B)  {$x$};
	\node at (2, -2) (C)  {$y$};
	\node at (0, -4) (D)  {$0$};
	
	\path 
	(A)  edge (B) 
	(A)  edge (C)
	(B)  edge (D)
	(C)  edge (D);
	\end{tikzpicture}
	\caption{}
	\label{wise-example}
\end{figure}

\begin{itemize}
	\item $(x \to 0) = \max \{t \mid t \cdot x \sqsubseteq 0 \} = y$
	\item $(x \to 1) = \max \{t \mid t \cdot x \sqsubseteq 1 \} = 1$
	\item $(x \to x) = 1$	- на что x ни умножай, он меньше либо равен самого себя
	\item $(1 \to x) = \max \{t \mid t \sqsubseteq x \} = x$
	\item $(0 \to x) = \max \{t \mid 0 \cdot t \sqsubseteq x \} = 1$	
\end{itemize}
Обратите внимание на первый пункт. Почему такой ответ? Потому что подходят $y$ (т.к $y \cdot x = 0, 0 \sqsubseteq 0$) и $0$, при этом $0 \sqsubseteq y $. Осознав это можно, интуитивно понять почему решётка, содержащая диамант, недистрибутивна.

\begin{theorem}
	Решётка c определенной операцией псевдодополнения "--- дистрибутивная решётка. То есть явно выписанное требование дистрибутивности в определении импликативной решетки было избыточным.
\end{theorem}

\begin{definition} 
	Единица (1) в решетке "--- наибольший элемент. 
	Ноль (0) в решетке "--- наименьший элемент. 
\end{definition}

\begin{theorem}
	В импликативной решетке есть 1.
\end{theorem}
\begin{proof}
	Пусть решётка не пуста. Рассмотрим произвольный элемент $k$. 
	\begin{gather*}
	k \to k = 1 \\
	1 = \max \{t \mid k \cdot t \sqsubseteq k \}
	\end{gather*}
	
	Но любой элемент решетки таков, что  $k \cdot t \sqsubseteq k$ (определение $\cdot$).
	
	Значит, 1 - наибольший элемент среди всех, что есть в решетке.
\end{proof}


\section{Псевдобулева алгебра}
\begin{definition}
	\emph{Псевдобулева алгебра} или \emph{алгебра Гейтинга} "--- это импликативная решётка с $0$-ём.
\end{definition}
В псевдобулевой алгебре можно рассмотреть \emph{операцию отрицания}, которая определяется, как псевдодополнение до нуля.
\[\sim a \equiv	a \to 0\]

\begin{definition}
	\emph{Булева алгебра} "--- это такая псевдобулева алгебра, что $ a + \sim a = 1$ при любом $a$.
\end{definition}
Пример:  $\langle{}X, \Omega \rangle$ - топологическое пространство. $(\Omega, \leq)$ является псевдобулевой алгеброй. \\
Утверждение: не все псевдобулевы алгебры "--- булевы. \\
Пример: $\mathbb{R}$ с классической топологией (Евклидово пространство):
\[(-\infty,0) \to \varnothing = (0, +\infty)\]
Но: \[(-\infty,0) \cup  (0, +\infty) \ne \mathbb{R}\]

\begin{definition} Оценка для алгебры Гейтинга $\langle{}X, \sqsubseteq \rangle$ : 
	
	Будем считать, что $\vdash \alpha$, если $ \llbracket \alpha \rrbracket = 1$ при любой оценке переменных. 
	
	Определим смысл связок: 
	\begin{gather*}
	\llbracket \alpha \with \beta \rrbracket = \llbracket \alpha \rrbracket \cdot \llbracket \beta \rrbracket \\
	\llbracket \alpha \vee \beta \rrbracket = \llbracket \alpha \rrbracket + \llbracket \beta \rrbracket \\
	\llbracket \alpha \rightarrow \beta \rrbracket = \llbracket \alpha \rrbracket \rightarrow \llbracket \beta \rrbracket \\
	\llbracket \neg \alpha \rrbracket = \thinspace \sim \llbracket \alpha \rrbracket 
	\end{gather*}

\end{definition}

\begin{theorem}
	Так определенная оценка корректна.
\end{theorem}
\begin{proof}
	Д/З.
\end{proof}
\begin{theorem}
	$ \nvdash_{intuitionistic} \alpha \vee \neg \alpha$.
\end{theorem}
\begin{proof} [Доказательство (модельное)] 
	Возьмём алгебру Гейтинга построеннную на $\mathbb{R}$. 
	
	Возьмём $\llbracket \alpha \rrbracket : = (-\infty, 0)$. 
	
	Тогда $\llbracket \alpha \vee \neg \alpha \rrbracket = \mathbb{R} \setminus \{0\} \neq \mathbb{R} $. 
	
	Т.е. $\nvDash \alpha \vee \neg \alpha$. 
	
	Значит, по теореме о корректности $ \nvdash_{intuitionistic} \alpha \vee \neg \alpha$.
\end{proof}

\begin{definition}
	$B$ - булева алгебра, $\alpha$ - классическое исчисление высказываний \\
	$ \vDash \alpha$, если $ \llbracket \alpha \rrbracket = 1$ при любой оценки переменных.
\end{definition}

\begin{theorem}
	Если $\vdash_{classic} \alpha$, то 	$ \vDash \alpha$.
\end{theorem}