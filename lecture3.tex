\section{Теорема о дедукции}

Соглашение об обозначениях. Будем обозначать буквами 
$\Gamma, \Delta, \Sigma, \Pi$ списки формул (возможно, пустые).

Пусть $\Gamma$ -- некоторый список высказываний, а $\alpha$ --- 
некоторое высказывание. 
Тогда мы будем говорить, что высказывание $\alpha$ \emph{выводимо} из $\Gamma$ 
(и записывать это как $\Gamma \vdash \alpha$), если существует такая 
последовательность высказываний $\alpha_1, \alpha_2, \dots \alpha_{n-1}, \alpha$
(называемая \emph{выводом} $\alpha$ из $\Gamma$), 
что каждое из высказываний $\alpha_i$ --- это 
\begin{itemize}
\item либо аксиома,
\vspace{-0.2cm}
\item либо получается по правилу Modus Ponens из предыдущих высказываний, 
\vspace{-0.2cm}
\item либо --- высказывание из списка $\Gamma$.
\end{itemize}
Элементы $\Gamma$ мы будем называть \emph{допущениями}. Также эти элементы
называют предположениями или гипотезами. 
\end{definition}

В свете данного определения можно заметить, что доказательство --- это
вывод из пустого списка допущений. 

\begin{theorem}\label{deduction_theorem}{Теорема о дедукции.}
Утверждение $\Gamma \vdash \alpha \rightarrow \beta$ справедливо тогда и 
только тогда, когда справедливо, что $\Gamma, \alpha \vdash \beta$.
\end{theorem}

Для доказательства рассмотрим следующую лемму:

\begin{lemma}
$\vdash \alpha \rightarrow \alpha$
\end{lemma}

\begin{proof}\ 

\begin{tabular}{lll}
(1) & $\alpha \rightarrow (\alpha \rightarrow \alpha)$&Сх. акс. 1\\
(2) & $(\alpha \rightarrow (\alpha \rightarrow \alpha)) \rightarrow 
  (\alpha \rightarrow ((\alpha \rightarrow \alpha) \rightarrow \alpha)) \rightarrow
  (\alpha \rightarrow \alpha)$&Сх. акс. 2\\
(3) & $(\alpha \rightarrow ((\alpha \rightarrow \alpha) \rightarrow \alpha)) \rightarrow
  (\alpha \rightarrow \alpha)$&M.P. 1,2\\
(4) & $(\alpha \rightarrow ((\alpha \rightarrow \alpha) \rightarrow \alpha))$ & Сх. акс. 1\\
(5) & $\alpha \rightarrow \alpha$ & M.P. 4,3\\
\end{tabular}\ 

\end{proof}

\begin{proof}[Доказательство теоремы \ref{deduction_theorem}]

Сперва докажем прямое следствие. Для этого нам 
достаточно научиться по любому выводу $\alpha \rightarrow \beta$ из $\Gamma$ строить 
вывод $\beta$ из $\Gamma, \alpha$. Возьмем вывод 
формулы $\alpha \rightarrow \beta$, то есть некоторую последовательность 
формул $\delta_1 \dots \delta_{m-1}; \alpha \rightarrow \beta$.
Добавив к выводу 2 формулы, получаем требуемый вывод:

\vspace{0.2cm}
\begin{tabular}{lll}
$(1)$ & $\delta_1$\\
& \dots\\
$(m-1)$ & $\delta_{m-1}$\\
$(m)$ & $\alpha \rightarrow \beta$\\
$(m+1)$ & $\alpha$ & <<Свежедобавленная>> аксиома \\
$(m+2)$ & $\beta$ & M.P. $m$, $m+1$\\
\end{tabular}
\vspace{0.2cm}

Теперь докажем обратное. Нам необходимо построить вывод утверждения 
$\Gamma \vdash \alpha \rightarrow \beta$ по имеющемуся выводу 
$\delta_1 \dots \delta_{m-1}, \beta$.
Мы поступим так: сперва набросаем план вывода -- разместим по тексту 
<<ключевые>> формулы, которые потом дополним до полноценного вывода 
промежуточными формулами. 

План вывода будет такой:

\begin{tabular}{ll}
$(1)$ & $\Gamma \vdash \alpha \rightarrow \delta_1$\\
\dots\\
$(m-1)$ & $\Gamma \vdash \alpha \rightarrow \delta_{m-1}$\\
$(m)$ & $\Gamma \vdash \alpha \rightarrow \beta$
\end{tabular}

Теперь надо дополнить его до полноценного вывода. Будем рассматривать
формулы подряд и перед каждой формулой добавлять некоторое количество
формул, обосновывающих соответствующий шаг доказательства. 
Рассмотрим формулу номер $i$. Возможны следующие варианты:

\begin{enumerate}
\item $\delta_i$ --- это аксиома или предположение, входящее в $\Gamma$. 
Тогда перед этой формулой вставим 
формулы $\delta_i$ и $\delta_i \rightarrow (\alpha \rightarrow \delta_i)$, 
и окажется, что $i$-я формула выводится из предыдущих двух формул
путем применения правила Modus Ponens.

\item $\delta_i$ совпадает с $\alpha$. Тогда мы вставим перед 
ней 4 первые формулы из леммы, и $\delta_i \rightarrow \alpha$ будет получаться
по правилу Modus Ponens.

\item $\delta_i$ выводится по правилу Modus Ponens из каких-то других
утверждений $\delta_j$ и $\delta_k$ (при этом 
$\delta_k \equiv \delta_j \rightarrow \delta_i$), где $j < i$ и $k < i$.
Покажем, что $\alpha \rightarrow \delta_i$ тоже может быть выведена
из утверждений $\alpha \rightarrow \delta_j$ и 
$\alpha \rightarrow (\delta_j \rightarrow \delta_i)$.

Для этого добавим два высказывания:

\begin{tabular}{lll}
$(\alpha \rightarrow \delta_j) \rightarrow ((\alpha \rightarrow (\delta_j \rightarrow \delta_i)) \rightarrow (\alpha \rightarrow \delta_i))$ & Сх. акс. 2\\
$((\alpha \rightarrow (\delta_j \rightarrow \delta_i)) \rightarrow (\alpha \rightarrow \delta_i))$ & M.P. из $j$ и $i-6$\\
\end{tabular}

\end{enumerate}
\end{proof}

По аналогии мы можем рассмотреть отношение \emph{следования}.
Будем говорить, что высказывание $\alpha$ следует из высказываний $\Gamma$,
если при любой оценке пропозициональных переменных, входящих в высказывания 
$\Gamma$ и $\alpha$, на которых все высказывания из $\Gamma$ истинны, 
$\alpha$ также истинно.
Записывать, что $\alpha$ следует из $\Gamma$, будем так: $\Gamma \models \alpha$.

\section{Теорема о полноте исчисления высказываний}

\begin{definition} 
Введем обозначение. Пусть $\alpha$ --- это некоторое высказывание, а $x$ --- некоторое
истинностное значение. Тогда обозначим за $_{[x]}\alpha$ высказывание
$\alpha$, если $x$ --- истина, и $\neg (\alpha)$, если $x$ --- ложь.
Также, если формула $\alpha$ --- это формула с $n$ пропозициональными переменными
$P_1 \dots P_n$, и $x_1 \dots x_n$ --- некоторые истинностные значения,
то за $\llbracket \alpha \rrbracket^{P_1 \coloneqq  x_1, \dots P_n \coloneqq  x_n}$ обозначим значение 
формулы $\alpha$ при подстановке значений $x_1 \dots x_n$ вместо переменных $P_1 \dots P_n$.
\end{definition}

\begin{lemma}\label{prefix_transposition}
Если $\Gamma, \Sigma \vdash \alpha$, то $\Gamma, \Delta, \Sigma \vdash \alpha$. 
Если $\Gamma, \Delta, \Sigma, \Pi \vdash \alpha$, то $\Gamma, \Sigma, \Delta, \Pi \vdash \alpha$. 
\end{lemma}

\begin{proof}Упражнение\end{proof}

\begin{lemma}
Если справедливы 3 утверждения: $\Gamma \vdash \gamma$, $\Delta \vdash \delta$ и
$\gamma, \delta \vdash \alpha$, то справедливо и $\Gamma, \Delta \vdash \alpha$
\end{lemma}

\begin{proof}
%Поскольку $\gamma, \delta \vdash \alpha$, то по теореме о дедукции найдется 
%доказательство $\gamma \rightarrow \delta \rightarrow \alpha$.

Мы получим требуемый вывод, просто последовательно соединив все три исходных 
вывода. Первые два вывода будут (очевидно) корретными при допущениях $\Gamma$ 
и $\Delta$. В третьем же выводе могут использоваться высказывания $\gamma$ и $\delta$,
отсутствующие в предположениях.
Но поскольку эти высказывания доказаны в первых двух частях вывода, мы будем
иметь полное право их упоминать --- на тех же основаниях, на которых они указаны
в конце соответствующих доказательств.

%Следующая последовательность высказываний является выводом $\alpha$ из
%$\Gamma$ и $\Delta$:
%
%\begin{tabular}{lll}
%$(1) \dots (m)$ & $\gamma_1 \dots \gamma_{m-1}, \gamma$ & Вывод $\Gamma \vdash \gamma$\\
%$(m+1) \dots (n)$ & $\delta_1 \dots \delta_{n-m-1}, \delta$ & Вывод $\Delta \vdash \delta$\\
%$(n+1) \dots (s)$ & $\sigma_1 \dots \sigma_{s-n-1}, \gamma \rightarrow \delta \rightarrow \alpha$ & Вывод $\gamma, \delta \vdash \alpha$\\
%$(s+1)$ & $\delta \rightarrow \alpha$ & M.P. $m$, $s$\\
%$(s+2)$ & $\alpha$ & M.P. $n$, $s+1$\\
%\end{tabular}

\end{proof}

Возьмем некоторую связку исчисления высказываний, например конъюнкцию: $A \with B$.
Построим для нее таблицу истинности. По каждой строчке построим утверждение,
в котором отрицания появляются там, где в таблице истинности находится \emph{Л}:

\begin{tabular}{cccc}\label{valuation_formula}
$A$ & $B$ & $A \with B$ & утверждение\\
\hline
Л & Л & Л & $\neg A, \neg B \vdash \neg (A \with B)$\\
Л & И & Л & $\neg A, B \vdash \neg (A \with B)$\\
И & Л & Л & $A, \neg B \vdash \neg (A \with B)$\\
И & И & И & $A, B \vdash A \with B$
\end{tabular}

\begin{lemma}
Каждое из построенных по таблицам истинности утверждений доказуемо.
\end{lemma}
\begin{proof} Упражнение.
\end{proof}

\begin{lemma}[Правило контрапозиции]\label{contraposition}Каковы бы ни были формулы $\alpha$ и $\beta$, справедливо, что 
$\vdash (\alpha \rightarrow \beta) \rightarrow (\neg\beta \rightarrow \neg\alpha)$
\end{lemma}

\begin{proof}
Сперва докажем, что $\alpha \rightarrow \beta, \neg\beta \vdash \neg\alpha$.

\begin{tabular}{lll}
(1) & $(\alpha \rightarrow \beta) \rightarrow (\alpha \rightarrow \neg\beta) \rightarrow \neg\alpha$ & Сх. акс. 9\\
(2) & $\alpha \rightarrow \beta$ & Допущение\\
(3) & $(\alpha \rightarrow \neg\beta) \rightarrow \neg\alpha$ & M.P. 2,1\\
(4) & $\neg\beta \rightarrow (\alpha \rightarrow \neg\beta)$ & Сх. акс. 1\\
(5) & $\neg\beta$ & Допущение\\
(6) & $\alpha \rightarrow \neg\beta$ & M.P. 5,4\\
(7) & $\neg\alpha$ & M.P. 6,3
\end{tabular}

%\begin{tabular}{lll}
%(1) & $\alpha \rightarrow \neg\neg\beta, \alpha \vdash \alpha \rightarrow \neg\neg\beta$
%\end{tabular}

Тогда, применив 2 раза Теорему о дедукции, получим вывод требуемого утверждения.
\end{proof}

\begin{lemma}{Правило исключенного третьего.}\label{excluded_third} Какова бы ни была формула $\alpha$,
$\vdash \alpha\vee\neg\alpha$
\end{lemma}
\begin{proof}

Доказательство проведем в 3 этапа.
\begin{enumerate}
\item Для начала покажем $\vdash \neg(\alpha\vee\neg\alpha) \rightarrow \neg\alpha$:

\begin{tabular}{lll}
$(1)$ & $\alpha \rightarrow \alpha \vee \neg\alpha$ & Сх. акс. 6\\
%    & доказательство из леммы \ref{contraposition} &\\
$(2)\dots(n+1)$ & $\gamma_1,\dots\gamma_{n-1},(\alpha \rightarrow \alpha \vee \neg\alpha) \rightarrow (\neg(\alpha\vee\neg\alpha) \rightarrow \neg\alpha)$ & Д-во из леммы \ref{contraposition}\\
$(n+2)$ & $\neg(\alpha\vee\neg\alpha) \rightarrow \neg\alpha$ & M.P. 1,$n+1$
\end{tabular}

\item
Затем докажем $\vdash \neg(\alpha\vee\neg\alpha) \rightarrow \neg\neg\alpha$:

\begin{tabular}{lll}
$(1)$ & $\neg\alpha \rightarrow \alpha \vee \neg\alpha$ & Сх. акс. 7\\
%    & доказательство из леммы \ref{contraposition} & \\
$(2)\dots(k+1)$ & $\delta_1, \dots \delta_{k-1}, (\neg\alpha \rightarrow \alpha\vee\neg\alpha) \rightarrow (\neg(\alpha\vee\neg\alpha) \rightarrow \neg\neg\alpha)$ & Д-во из леммы \ref{contraposition}\\
$(k+2)$ & $\neg(\alpha\vee\neg\alpha) \rightarrow \neg\neg\alpha$ & M.P. 1,$k+1$\\
\end{tabular}

%Применив теорему о дедукции, получим: $\neg(\alpha\vee\neg\alpha) \vdash \neg\neg\alpha$. Далее,
%в данном допущении, построим следующий вывод:
%
%\begin{tabular}{lll}
%$(1)$ & $\neg\neg\alpha \rightarrow \alpha$ & Сх. акс. 10\\
%$(2)\dots(l+1)$ & $\epsilon_1, \dots \epsilon_{l-1}, \neg\neg\alpha $ & По теореме о дедукции \\
%$(l+2)$ & $\alpha $ & M.P. $l+1$,1
%\end{tabular}
%
%И по теореме о дедукции из данного вывода мы можем получить доказательство
%утверждения $\vdash \neg(\alpha\vee\neg\alpha) \rightarrow \alpha$.

\item
Теперь докажем все вместе:

\begin{tabular}{lll}
%(1) & $\alpha \rightarrow \alpha \vee \neg\alpha$ & Сх. акс. 5\\
%    & доказательство из леммы \ref{contraposition} & 
%(2) & $(\alpha \rightarrow \alpha \vee \neg\alpha) \rightarrow (\neg(\alpha\vee\neg\alpha) \rightarrow \neg\alpha)$ & по лемме \ref{contraposition}\\
%    & \dots сюда вставим доказательство из пункта 1 \dots\\
(1) & $\neg(\alpha\vee\neg\alpha) \rightarrow \neg\alpha$ & по пункту 1\\
%(4) & $\neg\alpha \rightarrow \alpha \vee \neg\alpha$ & Сх. акс. 6\\
%    & доказательство из леммы \ref{contraposition} & \\
%(5) & $(\neg\alpha \rightarrow \alpha\vee\neg\alpha) \rightarrow (\neg(\alpha\vee\neg\alpha) \rightarrow \neg\neg\alpha) & по лемме \ref{contraposition}\\
%(2) & $\neg(\alpha\vee\neg\alpha) \rightarrow \neg\neg\alpha$ \\
%(7) & $(\neg\neg\alpha \rightarrow \alpha) \rightarrow \neg(\alpha\vee\neg\alpha) \rightarrow (\neg\neg\alpha \rightarrow \alpha)$ & Сх. акс. 1\\
%(8) & $(\neg\neg\alpha \rightarrow \alpha) & Сх. акс. 10\\
%(9) & $\neg(\alpha\vee\neg\alpha) \rightarrow (\neg\neg\alpha \rightarrow \alpha)$ & M.P. 8,7\\
%(10)& $(\neg(\alpha\vee\neg\alpha) \rightarrow \neg\neg\alpha) \rightarrow (\neg(\alpha\vee\neg\alpha) \rightarrow (\neg\neg\alpha \rightarrow \alpha)) \rightarrow \neg(\alpha\vee\neg\alpha \rightarrow \alpha)$
%(11) & $(\neg(\alpha\vee\neg\alpha) \rightarrow (\neg\neg\alpha \rightarrow \alpha)) \rightarrow \neg(\alpha\vee\neg\alpha \rightarrow \alpha)$ M.P. 6,10\\
%(12) & $\neg(\alpha\vee\neg\alpha \rightarrow \alpha)$
%    & по обратной т. к т. о дедукции из (6) можно получить &\\
%    & $\neg(\alpha\vee\neg\alpha) \vdash \neg\neg\alhpa$ &\\
%    & а затем, применив акс. (10), modus ponens и т. о дедукции: &\\
%    & \dots сюда вставим доказательство из пункта 2 dots\\
(2) & $\neg(\alpha\vee\neg\alpha) \rightarrow \neg\neg\alpha$& по пункту 2\\
(3) & $(\neg(\alpha\vee\neg\alpha) \rightarrow \neg\alpha) \rightarrow (\neg(\alpha\vee\neg\alpha) \rightarrow \neg\neg\alpha) \rightarrow (\neg\neg(\alpha\vee\neg\alpha))$ & Сх. акс. 9\\
(4) & $(\neg(\alpha\vee\neg\alpha) \rightarrow \neg\neg\alpha) \rightarrow \neg\neg(\alpha\vee\neg\alpha)$ & M.P. 1,3\\
(5) & $\neg\neg(\alpha\vee\neg\alpha)$ & M.P. 2,4\\
(6) & $\neg\neg(\alpha\vee\neg\alpha) \rightarrow (\alpha\vee\neg\alpha)$ & Сх. акс. 10\\
(7) & $\alpha\vee\neg\alpha$ & M.P. 5,6
\end{tabular}
\end{enumerate}

\end{proof}

\begin{lemma}{Об исключении допущения.}
Пусть справедливо $\Gamma, \rho \vdash \alpha$ и $\Gamma, \neg \rho \vdash \alpha$.
Тогда также справедливо $\Gamma \vdash \alpha$.
\end{lemma}

\begin{proof}

Применив теорему о дедукции к условиям теоремы получим следующее:

\begin{tabular}{l}
$\Gamma \vdash \rho \rightarrow \alpha$\\
$\Gamma \vdash \neg \rho \rightarrow \alpha$
\end{tabular}

Тогда следующий вывод покажет $\Gamma \vdash \alpha$:

\begin{tabular}{lll}
$(1) \dots (p)$ & $\gamma_1, \dots \gamma_{p-1}, \rho \rightarrow \alpha$ & Вывод $\Gamma \vdash \rho \rightarrow \alpha$\\
$(p+1) \dots (q)$ & $\delta_1, \dots \delta_{q-p-1}, \neg \rho \rightarrow \alpha$ & Вывод $\Gamma \vdash \neg \rho \rightarrow \alpha$\\
$(q+1) \dots (r)$ & $\epsilon_1, \dots \epsilon_{r-q-1}, \rho \vee \neg \rho$ & Лемма \ref{excluded_third}\\
$(r+1)$ & $(\rho \rightarrow \alpha) \rightarrow (\neg \rho \rightarrow \alpha) \rightarrow (\rho \vee \neg \rho) \rightarrow \alpha$ & Сх. аксиом 8\\
$(r+2)$ & $(\neg \rho \rightarrow \alpha) \rightarrow (\rho \vee \neg \rho \rightarrow \alpha)$ & M.P. $1$, $r+1$\\
$(r+3)$ & $\rho \vee \neg \rho \rightarrow \alpha$ & M.P. $p$, $r+2$\\
$(r+4)$ & $\alpha$ & M.P. $r$, $r+3$
\end{tabular}

\end{proof}

\begin{theorem}{О полноте исчисления высказываний.}
Пусть справедливо $\models \alpha$. Тогда также справедливо, что $\vdash \alpha$.
\end{theorem}

\begin{proof}
Для доказательства теоремы мы докажем чуть более сильное утверждение --- что 
для любого $k$ от $0$ до $n$ и любой
оценки переменных $x_1, \dots x_k$ справедливо 
$_{[x_1]} P_1, \dots _{[x_k]} P_k \vdash \alpha$.
Нетрудно заметить, что утверждение теоремы непосредственно следует из данного 
утверждения для $k=0$. 
Доказательство будет вестись индукцией по $n-k$.

База. Пусть $n-k=0$, то есть $k=n$. 
$\models \alpha$ означает, что при любой оценке 
$x_1, \dots x_n$ пропозициональных переменных $P_1, \dots P_n$
справедливо $\alpha [P_1 \coloneqq  x_1, \dots P_n \coloneqq  x_n] = \texttt{И}$.
Возьмем некоторую оценку переменных
$x_1, \dots x_n$. Тогда, по лемме \ref{valuation_formula},
$_{[x_1]} P_1, \dots _{[x_{n}]} P_{n} \vdash {_{\alpha [P_1 \coloneqq  x_1, \dots P_n \coloneqq  x_n]} \alpha}$
то есть $_{[x_1]} P_1, \dots _{[x_{n}]} P_{n} \vdash \alpha$.

Переход. Пусть утверждение уже доказано для некоторого $n-k > 0$, покажем его
для $n-k+1$ (то есть доказано для $k < n$, покажем его для $k-1$).
Возьмем некоторую оценку переменных $x_1, \dots x_{k-1}$.
По предположению, $_{[x_1]} P_1, \dots _{[x_{k}]} P_{k} \vdash \alpha$, то есть

\begin{tabular}{rl}
$_{[x_1]} P_1, \dots _{[x_{k-1}]} P_{k-1}, \neg P_{k}$ & $\vdash \alpha$\\
$_{[x_1]} P_1, \dots _{[x_{k-1}]} P_{k-1}, P_{k}$ & $\vdash \alpha$
\end{tabular}

Тогда по лемме об исключении допущения, справедливо 
$_{[x_1]} P_1, \dots _{[x_{k-1}]} P_{k-1} \vdash \alpha$.

\end{proof}

%\begin{lemma}\label{formula_valuation}
%Пусть $\alpha$ --- высказывание от $n$ пропозициональных переменных
%$P_1 \dots P_n$ и пусть выбрана некоторая их оценка $x_1 \dots x_n$.
%Тогда если $_{[x_1]}P_1 \dots _{[x_n]}P_n \vdash _{[y]} \alpha$, то $y = \alpha [x_1, \dots x_n]$
%\end{lemma}
%
%\begin{proof}
%Будем доказывать утверждение индукцией по структуре.
%
%База. Пусть $\alpha$ --- некоторая пропозициональная переменная $P_1$.
%Тогда по условию теоремы $_{[x_1]}P_1 \vdash _{[y]}P_1$. Ясно, что так как
%$P_1 [x_1] = x_1$, то $x = y$.
%\end{proof}

\begin{theorem}{О корректности исчисления высказываний.}
Пусть справедливо $\vdash \alpha$. Тогда также справедливо, что $\models \alpha$.
\end{theorem}

\begin{proof}
По условию теоремы, у нас есть доказательство высказывания $\alpha$, то есть 
последовательность высказываний $\alpha_1, \dots \alpha_m$.
Каждое высказывание --- это либо аксиома, либо применение правила Modus Ponens.
Докажем, что для каждого $k$ все высказывания $\alpha_l$ при $l \le k$ --- тавтологии.
Доказательство будем вести индукцией по $k$.

База. Пусть $k=0$, тогда нет ни одного высказывания, про которое нужно доказать, что
оно --- тавтология, то есть утверждение автоматически верно.

Переход. Пусть для некоторого $k$ утверждение справедливо, докажем его для $k+1$.
Выберем некоторую оценку $x_1, \dots x_n$ пропозициональных 
переменных $P_1, \dots P_n$, использованных в высказываниях $\alpha_1 \dots \alpha_{k+1}$.
Рассмотрим случаи. 

Пусть $\alpha_{k+1}$ --- аксиома. В данную аксиому входят одна, две или три формулы
$\beta_1, \beta_2, \beta_3$. Подставив всех возможных истинностных значений
вместо данных формул можно проверить, что все аксиомы являются тавтологиями,
значит, они будут истинны и на тех конкретных значениях, которые примут данные формулы
после подстановки значений $x_1, \dots x_n$.

Пусть $\alpha_{k+1}$ получается по правилу Modus Ponens из $\alpha_p$ и 
$\alpha_q$, причем $\alpha_q \equiv \alpha_p \rightarrow \alpha_{k+1}$.
Тогда $\llbracket\alpha_p\rrbracket^{P_1 \coloneqq  x_1, \dots P_n\coloneqq x_n} = \texttt{И}$ и 
$\llbracket\alpha_p \rightarrow \alpha_{k+1}\rrbracket^{P_1\coloneqq x_1, \dots P_n\coloneqq x_n} = \texttt{И}$.
Из таблицы истинности импликации следует, что неизбежно 
$\llbracket\alpha_{k+1}\rrbracket^{P_1\coloneqq x_1, \dots P_n\coloneqq x_n} = \texttt{И}$.

\end{proof}

Заметим, что вместе из этих двух теорем следует, что если неверно, что $\vdash \alpha$,
то неизбежно найдется контрпример.

%
%$\alpha [P_1 \coloneqq  \texttt{И}] = \texttt{И}]$,
%и $\alpha [P_1 \coloneqq  \texttt{Л}] = \texttt{И}]$.
%Тогда, по лемме ..., справедливо $P_1 \vdash \alpha$ и $\neg P_1 \vdash \alpha$.
%Тогда, по лемме об исключении предположения, справедливо и $\vdash \alpha$.
%
%Переход. Пусть для любой формулы от $n-1$ переменной из $\models \alpha$
%следует $\vdash \alpha$. Рассмотрим некоторую формулу от $n$ переменных $\alpha$,
%с пропозициональными переменными $P_1, \dots P_n$.
%Тогда (по лемме ...) справедливо следующее:
%
%Тогда, если при некотором $k$ и при любой оценке $x_1, \dots x_k$
%переменных $P_1, \dots P_k$
%выполнено $\alpha [P_1 \coloneqq  x_1, \dots P_k \coloneqq  x_k] = \texttt{И}$, то 
%Тогда (по лемме ...) справедливо следующее:
%
%См. Стефан Клини, Математическая логика, М. <<Мир>>, 1973. Глава 1, параграф 12, стр. 61-64.
%
%Данное доказательство изложено не до конца формально, но дальнейшая его формализация
%усложнит изложение, при этом ничего не даст для понимания вопроса. Надеемся, что каждый
%без труда восстановит математическую индукцию, прячющуюся за данным рассуждением.
%Также заметим, что поскольку данное доказательство изложено на мета-языке, эта апелляция
%к интуиции и здравому смыслу не должна нас чрезмерно смущать.
%\end{proof}


\section{Интуиционистское исчисление высказываний}

Заменим аксиому устранения двойного отрицания
на $\phi \rightarrow \neg \phi \rightarrow \psi$.
Полученная система называется интуиционистским исчислением высказываний.

\begin{theorem}\label{no_third_excluded}
В интуиционистском исчислении высказываний невозможно доказать правило 
исключенного третьего: $P \vee \neg P$.
\end{theorem}

Мы рассмотрим \emph{модельное} доказательство. То есть, мы построим некоторую модель для
исчисления высказываний --- так определим истинностные значения и связки,
что в нем правило исключенного третьего перестанет быть тавтологией, при этом построенная
нами модель останется корректной. 

Пусть множество истинностных значений будет состоять из трех элементов: \texttt{И},
\texttt{Н} и \texttt{Л}. Определим отношение порядка, будем считать, что 
$\texttt{И} > \texttt{Н} > \texttt{Л}$.

Все связки на истине и лжи действуют традиционно, необходимо описать только их
поведение с элементом \texttt{Н}.
Для двух операций дадим естественные обобщения.
$A \with B = \min(A,B)$, $A \vee B = \max(A,B)$.
Однако, положим $\neg \texttt{Н} = \texttt{Л}$.
Также, операцию $A \rightarrow B$ положим $\neg A \vee B$, за исключением
$\texttt{Н} \rightarrow \texttt{Н} = \texttt{И}$

Будем считать, что высказывание истинно на некоторой оценке, если оно на ней принимает
значение \texttt{И}. Общезначимость же ($\models \alpha$) означает, что 
$\alpha[x_1, \dots x_n] = \texttt{И}$ для любой оценки переменных $x_1, \dots x_n$.

Нетрудно показать, что так определенная модель исчисления является корректной --- то есть
любое утверждение, имеющее доказательство, является тавтологией в смысле данной модели.

\begin{proof}[Доказательство теоремы \ref{no_third_excluded}]
Поскольку $\texttt{Н} \vee \neg \texttt{Н} = \texttt{Н}$, то в данной модели неверно 
$\models P \vee \neg P$. Значит, если бы существовало доказательство этого утверждения,
то рассмотренная модель была бы некорректной.
Значит, для интуиционистского исчисления высказываний неверно $\vdash P \vee \neg P$. 
\end{proof}

Однако, так построенная логика (с такой моделью) не является полной. Например, 
высказывание $\neg P \vee \neg\neg P$, не являющееся доказуемым в интуиционистской 
логике, тем не менее оказывается истинным. Сейчас мы рассмотрим другие модели,
обладающие свойством полноты.

\subsection{Топологическая интерпретация интуиционистского исчисления высказываний}

Пусть дано некоторое исчисление высказываний, для которого нам нужно
построить модель --- предложить способ оценки истинности выражений.
Начинаем мы с множества истинностных значений.
Возьмем в качестве этого множества все открытые множества некоторого
заранее выбранного топологического пространства.
Определим оценку для связок интуиционистского исчисления высказываний следующим образом:

\begin{tabular}{l}\\
$\llbracket A \with B \rrbracket = \llbracket A \rrbracket \cap \llbracket B \rrbracket$\\
$\llbracket A \vee B \rrbracket = \llbracket A \rrbracket \cup \llbracket B \rrbracket$\\
$\llbracket A \rightarrow B \rrbracket = (c\llbracket A \rrbracket \cup \llbracket B \rrbracket)^\circ$\\
$\llbracket \neg A \rrbracket = (c \llbracket A \rrbracket)^\circ$
\end{tabular}

Будем считать, что формула истинна в данной модели, если её значение оказалось равно
всему пространству. 

Например, возьмем в качестве пространства $\mathbb{R}$, и вычислим значение формулы $A \vee \neg A$ 
при $A$ равном $(0,1)$: $\llbracket A \vee \neg A \rrbracket = (0,1) \cup \llbracket \neg A \rrbracket = 
(0,1) \cup (c(0,1))^\circ = (0,1) \cup ((-\infty,0)\cup(1,\infty)) = (-\infty,0)\cup(0,1)\cup(1,\infty)$.
Нетрудно видеть, что данная формула оказалась не общезначимой в данной интерпретации.

\subsection{Модели Крипке}

\begin{definition}
Рассмотрим некоторый ориентированный граф без циклов (без потери общности мы можем взять дерево вместо такого графа). 
Узлы мы назовем \emph{мирами} и пронумеруем натуральными числами: $W = \{ W_1, W_2, \dots, W_n\}$. 
Будем писать $W_i \preceq W_j$, если существует путь из $W_i$ в $W_j$.
Также мы считаем, что $W_i \preceq W_i$ (из каждой вершины существует путь в саму себя).
\end{definition}

\begin{definition}
Каждому узлу сопоставим множество \emph{вынужденных} переменных ИИВ и будем писать $W_i \Vdash A_k$, 
если переменная $A_k$ вынуждена в мире $W_i$. При этом, если $W_i \preceq W_j$, 
то всегда должно быть выполнено и $W_j \Vdash A_k$.
\end{definition}

Обобщим отношение вынужденности на случай произвольной формулы:
\begin{itemize}
\item Если $W_i \Vdash \alpha$ и $W_i \Vdash \beta$, то $W_i \Vdash \alpha\with\beta$;
\item Если $W_i \Vdash \alpha$ или $W_i \Vdash \beta$, то $W_i \Vdash \alpha\vee\beta$;
\item Если в любом мире $W_k: W_i \preceq W_k$ выполнено, что из $W_k \Vdash \alpha$ следует $W_k \Vdash \beta$, 
то $W_i \Vdash \alpha\rightarrow\beta$;
\item Если ни в каком мире $W_k: W_i \preceq W_k$ не выполнено $\alpha$, то $W_i \Vdash \neg\alpha$.
\end{itemize}

\begin{definition}
\emph{Моделью Крипке (шкалой Крипке)} назовем упорядоченную тройку "--- множество миров, отношение 
порядка на мирах и отношение вынужденности.
Будем говорить, что формула $\alpha$ \emph{вынуждается} моделью (или является истинной в данной модели),
если $W_i \Vdash \alpha$ в любом мире $W_i$. Будем записывать это как $\Vdash \alpha$.
\end{definition}

Философский смысл моделей Крипке можно объяснить следующим образом. Давайте понимать, 
что $W = \{ W_1, W_2, \dots, W_n\}$  есть множество возможных состояний мира (цивилизации); 
$W_i \preceq W_j$ означает, что мир $W_j$ может получиться 
из мира $W_i$ в результате развития цивилизации. Утверждение $W_i \Vdash \alpha$ означает, 
что в мире  $W_i$ установлено, что высказывание $\alpha$ истинно 
(при этом оно останется истинным и при дальнейшем развитии цивилизации). Истинность $\neg\alpha$ 
в мире $W_i$ означает, что ни при каком развитии цивилизации из 
состояния $W_i$ высказывание $\alpha$ не станет истинным.


\begin{lemma}{О монотонности.}
Если $W \Vdash \alpha$ и $W_i \preceq W_j$, то $W_j \Vdash \alpha$.
\end{lemma}
\begin{proof} 
Данное утверждение доказывается индукцией по построению формулы $\alpha$.
\end{proof}


\begin{theorem}{О корректности ИИВ относительно моделей Крипке.}
Если формула $\alpha$ выводима в интуиционистском исчислении высказываний, 
то она истинна во всех мирах всех моделей Крипке.
\end{theorem}
\begin{proof}[Доказательство.] Покажем, что все аксиомы истинны, а правило Modus Ponens сохраняет истинность. 
Второе очевидно верно: пусть во всех мирах $W_i$ формулы $\alpha$ и $\alpha \rightarrow \beta$ истинны, тогда из определения 
истинности импликации получаем, что $\beta$ так же истинна во всех мирах.

Теперь осталось показать, что все аксиомы истинны во всех мирах всех моделей Крипке. 
Будем пользоваться полезным фактом: для того, чтобы установить, что импликация $\alpha \rightarrow \beta$ истинна во всех мирах, 
надо проверить, что во всех  $W_i$, в которых вынужденна $\alpha$, также вынужденна и $\beta$ 
(если $\alpha$ не вынуждается, то $\beta$ может как вынуждаться, так и не вынуждаться). 

Рассмотрим первую схему аксиом: $\alpha \rightarrow \beta \rightarrow \alpha$. 
По лемме о монотонности, если $W_i \Vdash \alpha$, то для всех $W_j$ таких, что $W_i \preceq W_j$, $W_j \Vdash \alpha$.
А во всех таких мирах, в которых истинна $\alpha$, истинна и $\beta \rightarrow \alpha$.

Рассмотрим вторую схему аксиом: $(\alpha \rightarrow \beta) \rightarrow (\alpha \rightarrow \beta \rightarrow \gamma) \rightarrow (\alpha \rightarrow \gamma)$.
Мы должны доказать, что если $W_i \Vdash \alpha \rightarrow \beta$, то $W_i \Vdash (\alpha \rightarrow \beta \rightarrow \gamma) \rightarrow (\alpha \rightarrow \gamma)$. 
Докажем от противного. Пусть существует мир  $W_i$, такой, что в нем вторая аксиома является ложной. 
Тогда в нем должны быть вынуждены следующие формулы: $\alpha \rightarrow \beta$, $\alpha \rightarrow \beta \rightarrow \gamma$ и $\alpha$. 
Формула $\gamma$, в свою очередь, не должна быть вынужденной в мире $W_i$. 
Однако $W_i \Vdash \alpha$ и $W_i \Vdash \alpha \rightarrow \beta$, из чего следует  $W_k \Vdash \beta$.
во всех мирах $W_k$ таких, что  $W_i \preceq W_k$. При $W_i \Vdash \alpha \rightarrow \beta \rightarrow \gamma$ это означает справедливость
$W_j \Vdash \gamma$ во всех мирах $W_j$: $W_i \preceq $W_j$. Тогда из этого мы получили $W_i \Vdash \gamma$ "--- противоречие, а значит 
вторая схема аксиом истинна во всех мирах.
\\Аналогичным способ рассматриваются и последующие аксиомы.

\end{proof}




