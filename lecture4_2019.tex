\section{Алгебра Линденбаума}

В этой главе условимся считать, что мы работаем с 
интуционистским исчислением высказываний (ИИВ), $\alpha$ и $\beta$ - 
произвольные пропозициональные формулы.
Для задания алгебры Линденбаума дадим несколько определений:

\begin{definition}
$\alpha \sqsubseteq \beta$ (находится в отношении частичного порядка), 
если $\alpha \vdash \beta$
\end{definition}

\begin{definition}
$\alpha \approx \beta$ (эквивалентно), если $\alpha \sqsubseteq \beta$
и $\beta \sqsubseteq \alpha$
\end{definition}

Рассмотрим класс эквивалентности $\alpha$, обозначив его как $[\alpha]_{\approx}$

\begin{definition}
$[\alpha]_{\approx} \sqsubseteq [\beta]_{\approx}$, если $\alpha \sqsubseteq \beta$
\end{definition}

\begin{definition}
$[\alpha]_{\approx} = [\beta]_{\approx}$, если $\alpha \approx \beta$
\end{definition}


Теперь можем дать определение:
\begin{definition}
ИИВ с заданными на нём классами эквивалентности называется \emph{алгеброй Линденбаума}
(и обозначается как $\mathcal{L}$)
\end{definition}

\begin{lemma}
Алгебра Линденбаума является алгеброй Гейтинга.
\end{lemma}

\textbf{Пример.} ${\llbracket A \rrbracket}_{\mathcal{L}} = [A]$

\begin{consequence}
Алгебра Линденбаума полна как модель ИИВ.
\end{consequence}
\begin{proof}
$\models \alpha$, значит ${\llbracket \alpha \rrbracket}_{\mathcal{L}} = 1$.

Из примера выше: ${\llbracket \alpha \rrbracket}_{\mathcal{L}} = 
	[A \rightarrow A]_{\approx}$, т.е. $A \rightarrow A \sqsubseteq \alpha$.
	
С одной стороны получаем, что $A \rightarrow A \vdash \alpha$. С другой стороны мы 
знаем доказательство $\vdash A \rightarrow A$. Соединив вместе, получаем, что $\vdash \alpha$. Что и означает полноту алгебры Линденбаума.
\end{proof}

Теперь рассмотрим основные понятия, связанные с гёделевыми алгебрами.

\begin{definition}
Алгебра Гейтинга называется \emph{гёделевой}, если $\forall a, b: a + b = 1$
	влечёт $a = 1$ или $b = 1$.
\end{definition}

\begin{definition}
Исчисление дизъюнктивно, если из $\vdash \alpha \lor \beta$ следует $\vdash \alpha$
или $\vdash \beta$.
\end{definition}

\begin{definition}
\emph{Гёделевизация} алгебры $A$.


Пусть множество $A$ с отношением $\leq$ является алгеброй Гейтинга, 
и пусть $\omega \in A$. На множестве $A \cup {\omega}$
определим отношение $\leq$, положив для любых $a, b, \in A: a \leq \omega$,
если $a \neq 1$; $a \leq b \iff a \leq_{A} b$; $\omega \leq 1$. Проделанная операция есть гёделевизация
(Gödelization) алгебры $A$ и обозначается $\Gamma(A)$.
\end{definition}

\begin{lemma}
$\Gamma(A)$ - гёделева алгебра.
\end{lemma}

\begin{proof}
$1 = a + b$. Если $a < 1$ и $b < 1$, то $a + b \leq \omega$ - противоречие.
\end{proof}

\begin{lemma}
$\Gamma(A)$ - алгебра Гейтинга.
\end{lemma}

\begin{definition}

$f: \Gamma(A) \rightarrow A$

$f(1_{\Gamma(A)}) = f(\omega) = 1_{A}$

$f(X) = X$
\end{definition}

\begin{lemma}

$f(a + b) = f(a) + f(b)$

$f(a \cdot b) = f(a) \cdot f(b)$

$f(a \rightarrow b) = f(a) \rightarrow f(b)$

$f(1) = 1$
\end{lemma}

\begin{lemma}
${\llbracket \alpha \rrbracket}_{\Gamma(\mathcal{L})} \neq 1$, тогда и только тогда, когда 
${\llbracket \alpha \rrbracket}_{\mathcal{L}} \neq 1$
\end{lemma}

\begin{theorem}
Таким образом, алгебра Линденбаума - гёделева.
\end{theorem}
\begin{proof}
Пусть $\vdash \alpha \lor \beta$.

${\llbracket \alpha \rrbracket}_{\mathcal{L}} + {\llbracket \beta \rrbracket}_{\mathcal{L}} = 1_{\mathcal{L}}$.

${\llbracket \alpha \lor \beta \rrbracket}_{\mathcal{L}} = 1_{\mathcal{L}}$, а значит по предыдущей лемме

${\llbracket \alpha \lor \beta \rrbracket}_{\Gamma(\mathcal{L})} = 1_{\Gamma(\mathcal{L})}$.

Значит, ${\llbracket \alpha \rrbracket}_{\Gamma(\mathcal{L})} = 1_{\Gamma(\mathcal{L})}$ или
${\llbracket \beta \rrbracket}_{\Gamma(\mathcal{L})} = 1_{\Gamma(\mathcal{L})}$

Значит, $\vdash \alpha$ или $\vdash \beta$.
\end{proof}

\begin{consequence}
$\vdash \alpha \lor \beta$ влечёт $\vdash \alpha$ или $\vdash \beta$.
\end{consequence}

\section{Дизъюнктивность ИИВ}

\begin{lemma}
${\llbracket \alpha \rrbracket}_{\mathcal{L}} = 1_{\mathcal{L}}$
\end{lemma}

\begin{proof}
Если $\vdash \alpha$, то ${\llbracket \alpha \rrbracket}_{A} = 1_{A}$, где $A$ - алгебра Гейтинга.

То есть ${\llbracket \alpha \rrbracket}_{\mathcal{L}} = 1_{\mathcal{L}}$

${\llbracket \alpha \rrbracket}_{\Gamma(\mathcal{L})} = 1_{\Gamma(\mathcal{L})}$
\end{proof}

\begin{lemma}
Если ${\llbracket \alpha \rrbracket}_{\mathcal{L}} \neq 1$, то ${\llbracket \alpha \rrbracket}_{\Gamma(\mathcal{L})} \neq 1$
\end{lemma}

\begin{proof}
//TODO
\end{proof}

\begin{lemma}
Если ${\llbracket \alpha \rrbracket}_{\Gamma(\mathcal{L})} = 1$, то
\\TODO
\end{lemma}

Итак, если ${\llbracket \alpha \rrbracket}_{\Gamma(\mathcal{L})} = 1$ при любой оценке,
то и ${\llbracket \alpha \rrbracket}_{\mathcal{L}} = 1$ при любой оценке.

\begin{definition}
Пропозициональное исчисление называется табличным, если существует конечная точная модель этого исчисления, т.е.
такая конечная логическая матрица, в которой истинны те и
только те пропозициональные формулы, которые выводимы в
этом исчислении.
\end{definition}

1) ИИВ дизъюнктивно
2) ИИВ нетаблично